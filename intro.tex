%Introduction File
\documentclass[./main]{subfiles}
\setcounter{section}{1}
\begin{document}
%■■■■■ 第1章 ■■■■■■■■■■■■■■■■■■■■■■■■■■■■■■■■■■■■■■■■■■■■■■■■■■■■■■■■■■■■■■■■■■■
\section{緒言}
\label{sec: intro}
このファイルは,東京工業大学工学院機械系における学士特定課題研究の報告書に使用するテンプレートで,1行の文字数,1ページの行数を初めとして,ほとんどの書式を「スタイル」として登録してある.\\
本文の文字数は,1ページ当たり50文字×46行×1段組で2300字とする.また,文章の区切りには全角の読点「,」(カンマ)と句点「.」(ピリオド)を用いる.カッコも原則として全角とする.\\
本文中にはMS明朝とTimes New Romanを使用し,章節項の見出しなどにはMSゴシックとTimes New Roman(ボールド体)を使用する.\\
単位は\textit{l}
%==== 1.1 節 ===================================================================
\subsection{研究背景}
\label{subsec: intro-background}
\begin{equation}
  \bar{C}(t)=\frac{1}{N}\sum^N_{t=1}C_f(t)s
  \label{eq: siki}
\end{equation}
ここに参考文献の引用の仕方を書いておきます.
\Eq{siki}\citep{白井大地:2013-09}でのべたが先行研究\citep{nin}によると,\citep{matloff__2012}
%---- 1.1.1 節 -----------------------------------------------------------------
\subsection{できそう}
\label{subsec: intro-surgical_robot}


%---- 1.1.2 節 -----------------------------------------------------------------
\subsection{マスタコントローラ}
\label{subsec: intro-master_controller}



%---- 1.1.3 節 -----------------------------------------------------------------
\subsection{先行研究}
\label{subsec: intro-previous_research}



%==== 1.2 節 ===================================================================
\subsection{研究目的}
\label{subsec: intro-purpose}


%==== 1.3 節 ===================================================================
\subsection{本論文の構成}
\label{subsec: intro-contents}




\biblio
\end{document}